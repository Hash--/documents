
% Default to the notebook output style

    


% Inherit from the specified cell style.




    
\documentclass{article}

    
    
    \usepackage{graphicx} % Used to insert images
    \usepackage{adjustbox} % Used to constrain images to a maximum size 
    \usepackage{color} % Allow colors to be defined
    \usepackage{enumerate} % Needed for markdown enumerations to work
    \usepackage{geometry} % Used to adjust the document margins
    \usepackage{amsmath} % Equations
    \usepackage{amssymb} % Equations
    \usepackage[mathletters]{ucs} % Extended unicode (utf-8) support
    \usepackage[utf8x]{inputenc} % Allow utf-8 characters in the tex document
    \usepackage{fancyvrb} % verbatim replacement that allows latex
    \usepackage{grffile} % extends the file name processing of package graphics 
                         % to support a larger range 
    % The hyperref package gives us a pdf with properly built
    % internal navigation ('pdf bookmarks' for the table of contents,
    % internal cross-reference links, web links for URLs, etc.)
    \usepackage{hyperref}
    \usepackage{longtable} % longtable support required by pandoc >1.10
    \usepackage{booktabs}  % table support for pandoc > 1.12.2
    

    
    
    \definecolor{orange}{cmyk}{0,0.4,0.8,0.2}
    \definecolor{darkorange}{rgb}{.71,0.21,0.01}
    \definecolor{darkgreen}{rgb}{.12,.54,.11}
    \definecolor{myteal}{rgb}{.26, .44, .56}
    \definecolor{gray}{gray}{0.45}
    \definecolor{lightgray}{gray}{.95}
    \definecolor{mediumgray}{gray}{.8}
    \definecolor{inputbackground}{rgb}{.95, .95, .85}
    \definecolor{outputbackground}{rgb}{.95, .95, .95}
    \definecolor{traceback}{rgb}{1, .95, .95}
    % ansi colors
    \definecolor{red}{rgb}{.6,0,0}
    \definecolor{green}{rgb}{0,.65,0}
    \definecolor{brown}{rgb}{0.6,0.6,0}
    \definecolor{blue}{rgb}{0,.145,.698}
    \definecolor{purple}{rgb}{.698,.145,.698}
    \definecolor{cyan}{rgb}{0,.698,.698}
    \definecolor{lightgray}{gray}{0.5}
    
    % bright ansi colors
    \definecolor{darkgray}{gray}{0.25}
    \definecolor{lightred}{rgb}{1.0,0.39,0.28}
    \definecolor{lightgreen}{rgb}{0.48,0.99,0.0}
    \definecolor{lightblue}{rgb}{0.53,0.81,0.92}
    \definecolor{lightpurple}{rgb}{0.87,0.63,0.87}
    \definecolor{lightcyan}{rgb}{0.5,1.0,0.83}
    
    % commands and environments needed by pandoc snippets
    % extracted from the output of `pandoc -s`
    \DefineVerbatimEnvironment{Highlighting}{Verbatim}{commandchars=\\\{\}}
    % Add ',fontsize=\small' for more characters per line
    \newenvironment{Shaded}{}{}
    \newcommand{\KeywordTok}[1]{\textcolor[rgb]{0.00,0.44,0.13}{\textbf{{#1}}}}
    \newcommand{\DataTypeTok}[1]{\textcolor[rgb]{0.56,0.13,0.00}{{#1}}}
    \newcommand{\DecValTok}[1]{\textcolor[rgb]{0.25,0.63,0.44}{{#1}}}
    \newcommand{\BaseNTok}[1]{\textcolor[rgb]{0.25,0.63,0.44}{{#1}}}
    \newcommand{\FloatTok}[1]{\textcolor[rgb]{0.25,0.63,0.44}{{#1}}}
    \newcommand{\CharTok}[1]{\textcolor[rgb]{0.25,0.44,0.63}{{#1}}}
    \newcommand{\StringTok}[1]{\textcolor[rgb]{0.25,0.44,0.63}{{#1}}}
    \newcommand{\CommentTok}[1]{\textcolor[rgb]{0.38,0.63,0.69}{\textit{{#1}}}}
    \newcommand{\OtherTok}[1]{\textcolor[rgb]{0.00,0.44,0.13}{{#1}}}
    \newcommand{\AlertTok}[1]{\textcolor[rgb]{1.00,0.00,0.00}{\textbf{{#1}}}}
    \newcommand{\FunctionTok}[1]{\textcolor[rgb]{0.02,0.16,0.49}{{#1}}}
    \newcommand{\RegionMarkerTok}[1]{{#1}}
    \newcommand{\ErrorTok}[1]{\textcolor[rgb]{1.00,0.00,0.00}{\textbf{{#1}}}}
    \newcommand{\NormalTok}[1]{{#1}}
    
    % Define a nice break command that doesn't care if a line doesn't already
    % exist.
    \def\br{\hspace*{\fill} \\* }
    % Math Jax compatability definitions
    \def\gt{>}
    \def\lt{<}
    % Document parameters
    \title{IC-Hands-on\_solutions}
    
    
    

    % Pygments definitions
    
\makeatletter
\def\PY@reset{\let\PY@it=\relax \let\PY@bf=\relax%
    \let\PY@ul=\relax \let\PY@tc=\relax%
    \let\PY@bc=\relax \let\PY@ff=\relax}
\def\PY@tok#1{\csname PY@tok@#1\endcsname}
\def\PY@toks#1+{\ifx\relax#1\empty\else%
    \PY@tok{#1}\expandafter\PY@toks\fi}
\def\PY@do#1{\PY@bc{\PY@tc{\PY@ul{%
    \PY@it{\PY@bf{\PY@ff{#1}}}}}}}
\def\PY#1#2{\PY@reset\PY@toks#1+\relax+\PY@do{#2}}

\expandafter\def\csname PY@tok@gd\endcsname{\def\PY@tc##1{\textcolor[rgb]{0.63,0.00,0.00}{##1}}}
\expandafter\def\csname PY@tok@gu\endcsname{\let\PY@bf=\textbf\def\PY@tc##1{\textcolor[rgb]{0.50,0.00,0.50}{##1}}}
\expandafter\def\csname PY@tok@gt\endcsname{\def\PY@tc##1{\textcolor[rgb]{0.00,0.27,0.87}{##1}}}
\expandafter\def\csname PY@tok@gs\endcsname{\let\PY@bf=\textbf}
\expandafter\def\csname PY@tok@gr\endcsname{\def\PY@tc##1{\textcolor[rgb]{1.00,0.00,0.00}{##1}}}
\expandafter\def\csname PY@tok@cm\endcsname{\let\PY@it=\textit\def\PY@tc##1{\textcolor[rgb]{0.25,0.50,0.50}{##1}}}
\expandafter\def\csname PY@tok@vg\endcsname{\def\PY@tc##1{\textcolor[rgb]{0.10,0.09,0.49}{##1}}}
\expandafter\def\csname PY@tok@m\endcsname{\def\PY@tc##1{\textcolor[rgb]{0.40,0.40,0.40}{##1}}}
\expandafter\def\csname PY@tok@mh\endcsname{\def\PY@tc##1{\textcolor[rgb]{0.40,0.40,0.40}{##1}}}
\expandafter\def\csname PY@tok@go\endcsname{\def\PY@tc##1{\textcolor[rgb]{0.53,0.53,0.53}{##1}}}
\expandafter\def\csname PY@tok@ge\endcsname{\let\PY@it=\textit}
\expandafter\def\csname PY@tok@vc\endcsname{\def\PY@tc##1{\textcolor[rgb]{0.10,0.09,0.49}{##1}}}
\expandafter\def\csname PY@tok@il\endcsname{\def\PY@tc##1{\textcolor[rgb]{0.40,0.40,0.40}{##1}}}
\expandafter\def\csname PY@tok@cs\endcsname{\let\PY@it=\textit\def\PY@tc##1{\textcolor[rgb]{0.25,0.50,0.50}{##1}}}
\expandafter\def\csname PY@tok@cp\endcsname{\def\PY@tc##1{\textcolor[rgb]{0.74,0.48,0.00}{##1}}}
\expandafter\def\csname PY@tok@gi\endcsname{\def\PY@tc##1{\textcolor[rgb]{0.00,0.63,0.00}{##1}}}
\expandafter\def\csname PY@tok@gh\endcsname{\let\PY@bf=\textbf\def\PY@tc##1{\textcolor[rgb]{0.00,0.00,0.50}{##1}}}
\expandafter\def\csname PY@tok@ni\endcsname{\let\PY@bf=\textbf\def\PY@tc##1{\textcolor[rgb]{0.60,0.60,0.60}{##1}}}
\expandafter\def\csname PY@tok@nl\endcsname{\def\PY@tc##1{\textcolor[rgb]{0.63,0.63,0.00}{##1}}}
\expandafter\def\csname PY@tok@nn\endcsname{\let\PY@bf=\textbf\def\PY@tc##1{\textcolor[rgb]{0.00,0.00,1.00}{##1}}}
\expandafter\def\csname PY@tok@no\endcsname{\def\PY@tc##1{\textcolor[rgb]{0.53,0.00,0.00}{##1}}}
\expandafter\def\csname PY@tok@na\endcsname{\def\PY@tc##1{\textcolor[rgb]{0.49,0.56,0.16}{##1}}}
\expandafter\def\csname PY@tok@nb\endcsname{\def\PY@tc##1{\textcolor[rgb]{0.00,0.50,0.00}{##1}}}
\expandafter\def\csname PY@tok@nc\endcsname{\let\PY@bf=\textbf\def\PY@tc##1{\textcolor[rgb]{0.00,0.00,1.00}{##1}}}
\expandafter\def\csname PY@tok@nd\endcsname{\def\PY@tc##1{\textcolor[rgb]{0.67,0.13,1.00}{##1}}}
\expandafter\def\csname PY@tok@ne\endcsname{\let\PY@bf=\textbf\def\PY@tc##1{\textcolor[rgb]{0.82,0.25,0.23}{##1}}}
\expandafter\def\csname PY@tok@nf\endcsname{\def\PY@tc##1{\textcolor[rgb]{0.00,0.00,1.00}{##1}}}
\expandafter\def\csname PY@tok@si\endcsname{\let\PY@bf=\textbf\def\PY@tc##1{\textcolor[rgb]{0.73,0.40,0.53}{##1}}}
\expandafter\def\csname PY@tok@s2\endcsname{\def\PY@tc##1{\textcolor[rgb]{0.73,0.13,0.13}{##1}}}
\expandafter\def\csname PY@tok@vi\endcsname{\def\PY@tc##1{\textcolor[rgb]{0.10,0.09,0.49}{##1}}}
\expandafter\def\csname PY@tok@nt\endcsname{\let\PY@bf=\textbf\def\PY@tc##1{\textcolor[rgb]{0.00,0.50,0.00}{##1}}}
\expandafter\def\csname PY@tok@nv\endcsname{\def\PY@tc##1{\textcolor[rgb]{0.10,0.09,0.49}{##1}}}
\expandafter\def\csname PY@tok@s1\endcsname{\def\PY@tc##1{\textcolor[rgb]{0.73,0.13,0.13}{##1}}}
\expandafter\def\csname PY@tok@sh\endcsname{\def\PY@tc##1{\textcolor[rgb]{0.73,0.13,0.13}{##1}}}
\expandafter\def\csname PY@tok@sc\endcsname{\def\PY@tc##1{\textcolor[rgb]{0.73,0.13,0.13}{##1}}}
\expandafter\def\csname PY@tok@sx\endcsname{\def\PY@tc##1{\textcolor[rgb]{0.00,0.50,0.00}{##1}}}
\expandafter\def\csname PY@tok@bp\endcsname{\def\PY@tc##1{\textcolor[rgb]{0.00,0.50,0.00}{##1}}}
\expandafter\def\csname PY@tok@c1\endcsname{\let\PY@it=\textit\def\PY@tc##1{\textcolor[rgb]{0.25,0.50,0.50}{##1}}}
\expandafter\def\csname PY@tok@kc\endcsname{\let\PY@bf=\textbf\def\PY@tc##1{\textcolor[rgb]{0.00,0.50,0.00}{##1}}}
\expandafter\def\csname PY@tok@c\endcsname{\let\PY@it=\textit\def\PY@tc##1{\textcolor[rgb]{0.25,0.50,0.50}{##1}}}
\expandafter\def\csname PY@tok@mf\endcsname{\def\PY@tc##1{\textcolor[rgb]{0.40,0.40,0.40}{##1}}}
\expandafter\def\csname PY@tok@err\endcsname{\def\PY@bc##1{\setlength{\fboxsep}{0pt}\fcolorbox[rgb]{1.00,0.00,0.00}{1,1,1}{\strut ##1}}}
\expandafter\def\csname PY@tok@kd\endcsname{\let\PY@bf=\textbf\def\PY@tc##1{\textcolor[rgb]{0.00,0.50,0.00}{##1}}}
\expandafter\def\csname PY@tok@ss\endcsname{\def\PY@tc##1{\textcolor[rgb]{0.10,0.09,0.49}{##1}}}
\expandafter\def\csname PY@tok@sr\endcsname{\def\PY@tc##1{\textcolor[rgb]{0.73,0.40,0.53}{##1}}}
\expandafter\def\csname PY@tok@mo\endcsname{\def\PY@tc##1{\textcolor[rgb]{0.40,0.40,0.40}{##1}}}
\expandafter\def\csname PY@tok@kn\endcsname{\let\PY@bf=\textbf\def\PY@tc##1{\textcolor[rgb]{0.00,0.50,0.00}{##1}}}
\expandafter\def\csname PY@tok@mi\endcsname{\def\PY@tc##1{\textcolor[rgb]{0.40,0.40,0.40}{##1}}}
\expandafter\def\csname PY@tok@gp\endcsname{\let\PY@bf=\textbf\def\PY@tc##1{\textcolor[rgb]{0.00,0.00,0.50}{##1}}}
\expandafter\def\csname PY@tok@o\endcsname{\def\PY@tc##1{\textcolor[rgb]{0.40,0.40,0.40}{##1}}}
\expandafter\def\csname PY@tok@kr\endcsname{\let\PY@bf=\textbf\def\PY@tc##1{\textcolor[rgb]{0.00,0.50,0.00}{##1}}}
\expandafter\def\csname PY@tok@s\endcsname{\def\PY@tc##1{\textcolor[rgb]{0.73,0.13,0.13}{##1}}}
\expandafter\def\csname PY@tok@kp\endcsname{\def\PY@tc##1{\textcolor[rgb]{0.00,0.50,0.00}{##1}}}
\expandafter\def\csname PY@tok@w\endcsname{\def\PY@tc##1{\textcolor[rgb]{0.73,0.73,0.73}{##1}}}
\expandafter\def\csname PY@tok@kt\endcsname{\def\PY@tc##1{\textcolor[rgb]{0.69,0.00,0.25}{##1}}}
\expandafter\def\csname PY@tok@ow\endcsname{\let\PY@bf=\textbf\def\PY@tc##1{\textcolor[rgb]{0.67,0.13,1.00}{##1}}}
\expandafter\def\csname PY@tok@sb\endcsname{\def\PY@tc##1{\textcolor[rgb]{0.73,0.13,0.13}{##1}}}
\expandafter\def\csname PY@tok@k\endcsname{\let\PY@bf=\textbf\def\PY@tc##1{\textcolor[rgb]{0.00,0.50,0.00}{##1}}}
\expandafter\def\csname PY@tok@se\endcsname{\let\PY@bf=\textbf\def\PY@tc##1{\textcolor[rgb]{0.73,0.40,0.13}{##1}}}
\expandafter\def\csname PY@tok@sd\endcsname{\let\PY@it=\textit\def\PY@tc##1{\textcolor[rgb]{0.73,0.13,0.13}{##1}}}

\def\PYZbs{\char`\\}
\def\PYZus{\char`\_}
\def\PYZob{\char`\{}
\def\PYZcb{\char`\}}
\def\PYZca{\char`\^}
\def\PYZam{\char`\&}
\def\PYZlt{\char`\<}
\def\PYZgt{\char`\>}
\def\PYZsh{\char`\#}
\def\PYZpc{\char`\%}
\def\PYZdl{\char`\$}
\def\PYZhy{\char`\-}
\def\PYZsq{\char`\'}
\def\PYZdq{\char`\"}
\def\PYZti{\char`\~}
% for compatibility with earlier versions
\def\PYZat{@}
\def\PYZlb{[}
\def\PYZrb{]}
\makeatother


    % Exact colors from NB
    \definecolor{incolor}{rgb}{0.0, 0.0, 0.5}
    \definecolor{outcolor}{rgb}{0.545, 0.0, 0.0}



    
    % Prevent overflowing lines due to hard-to-break entities
    \sloppy 
    % Setup hyperref package
    \hypersetup{
      breaklinks=true,  % so long urls are correctly broken across lines
      colorlinks=true,
      urlcolor=blue,
      linkcolor=darkorange,
      citecolor=darkgreen,
      }
    % Slightly bigger margins than the latex defaults
    
    \geometry{verbose,tmargin=1in,bmargin=1in,lmargin=1in,rmargin=1in}
    
    

    \begin{document}
    
    
    \maketitle
    
    

    
    \begin{Verbatim}[commandchars=\\\{\}]
{\color{incolor}In [{\color{incolor}11}]:} \PY{k+kn}{from} \PY{n+nn}{scipy.constants} \PY{k+kn}{import} \PY{n}{mu\PYZus{}0}\PY{p}{,} \PY{n}{c}
\end{Verbatim}

    \section{Power Handling}

    For coaxial, the electric field is : \[
E(\rho)=\frac{1}{\rho}\frac{V}{\ln\left(b/a\right)}
\]

    where $a<\rho<b$.

    This is maximum

    \begin{Verbatim}[commandchars=\\\{\}]
{\color{incolor}In [{\color{incolor}1}]:} \PY{n}{a} \PY{o}{=} \PY{l+m+mf}{140e\PYZhy{}3} \PY{o}{/}\PY{l+m+mi}{2} \PY{c}{\PYZsh{} inner conductor radius}
        \PY{n}{b} \PY{o}{=} \PY{l+m+mf}{230e\PYZhy{}3} \PY{o}{/}\PY{l+m+mi}{2} \PY{c}{\PYZsh{} inner conductor radius}
\end{Verbatim}

    \begin{Verbatim}[commandchars=\\\{\}]
{\color{incolor}In [{\color{incolor}2}]:} \PY{k}{def} \PY{n+nf}{coax\PYZus{}electric\PYZus{}field}\PY{p}{(}\PY{n}{rho}\PY{p}{,} \PY{n}{V}\PY{p}{,} \PY{n}{a}\PY{p}{,} \PY{n}{b}\PY{p}{)}\PY{p}{:}
            \PY{l+s+sd}{\PYZdq{}\PYZdq{}\PYZdq{}}
        \PY{l+s+sd}{    Returns the electric field in a coaxial line.}
        \PY{l+s+sd}{    \PYZdq{}\PYZdq{}\PYZdq{}}
            \PY{k}{return} \PY{l+m+mi}{1}\PY{o}{/}\PY{n}{rho}\PY{o}{*}\PY{n}{V}\PY{o}{/}\PY{n}{log}\PY{p}{(}\PY{n}{b}\PY{o}{/}\PY{n}{a}\PY{p}{)}
\end{Verbatim}

    \begin{Verbatim}[commandchars=\\\{\}]
{\color{incolor}In [{\color{incolor}3}]:} \PY{n}{rho} \PY{o}{=} \PY{n}{linspace}\PY{p}{(}\PY{n}{a}\PY{p}{,} \PY{n}{b}\PY{p}{,} \PY{l+m+mi}{101}\PY{p}{)}
        \PY{n}{V} \PY{o}{=} \PY{l+m+mi}{1} \PY{c}{\PYZsh{} V}
        \PY{n}{plot}\PY{p}{(}\PY{n}{rho}\PY{o}{/}\PY{n}{a}\PY{p}{,} \PY{n}{coax\PYZus{}electric\PYZus{}field}\PY{p}{(}\PY{n}{rho}\PY{p}{,} \PY{n}{V}\PY{p}{,} \PY{n}{a}\PY{p}{,} \PY{n}{b}\PY{p}{)}\PY{o}{/}\PY{n}{V}\PY{p}{)}
        \PY{n}{xlabel}\PY{p}{(}\PY{l+s}{\PYZsq{}}\PY{l+s}{rho/a}\PY{l+s}{\PYZsq{}}\PY{p}{)}
        \PY{n}{ylabel}\PY{p}{(}\PY{l+s}{\PYZsq{}}\PY{l+s}{E\PYZus{}\PYZob{}rho\PYZcb{} / V}\PY{l+s}{\PYZsq{}}\PY{p}{)}
\end{Verbatim}

            \begin{Verbatim}[commandchars=\\\{\}]
{\color{outcolor}Out[{\color{outcolor}3}]:} <matplotlib.text.Text at 0x67e47b8>
\end{Verbatim}
        
    \begin{center}
    \adjustimage{max size={0.9\linewidth}{0.9\paperheight}}{IC-Hands-on_solutions_files/IC-Hands-on_solutions_7_1.png}
    \end{center}
    { \hspace*{\fill} \\}
    
    This is maximum for $\rho=a$. Let us suppose that the peak voltage $V_p$
(or equivalently the maximum RF power $P_p$ the line can handle) is set
by the breakdown voltage of the insulator. The electric field strength
at which breakdown occurs depends on the respective geometries of the
insulator and the electrodes with which the electric field is applied.
In the case of the air, the dielectric strength is $E_p$=3.0 MV/m. \[
E_{max} = E(\rho=a) = V_p / (a \ln(b/a))
\]

    Thus, \[
V_p = E_{max} a \ln(b/a)
\]

    \begin{Verbatim}[commandchars=\\\{\}]
{\color{incolor}In [{\color{incolor}4}]:} \PY{n}{Ep} \PY{o}{=} \PY{l+m+mf}{3.0e6} \PY{c}{\PYZsh{} V/m}
        \PY{n}{Vp} \PY{o}{=} \PY{n}{Ep}\PY{o}{*}\PY{n}{a}\PY{o}{*}\PY{n}{log}\PY{p}{(}\PY{n}{b}\PY{o}{/}\PY{n}{a}\PY{p}{)}
        \PY{n}{Vp}
\end{Verbatim}

            \begin{Verbatim}[commandchars=\\\{\}]
{\color{outcolor}Out[{\color{outcolor}4}]:} 104251.74612591714
\end{Verbatim}
        
    One get a peak voltage of 104kV. In terms of power, for a 30 Ohm
characteric impedance, this leads to: \[
P_d = \frac{V^2}{2 Z_0}
\]

    \begin{Verbatim}[commandchars=\\\{\}]
{\color{incolor}In [{\color{incolor}5}]:} \PY{n}{Z0} \PY{o}{=} \PY{l+m+mi}{30}
        \PY{n}{Pd} \PY{o}{=} \PY{n}{Vp}\PY{o}{*}\PY{o}{*}\PY{l+m+mi}{2} \PY{o}{/} \PY{p}{(}\PY{l+m+mi}{2}\PY{o}{*}\PY{n}{Z0}\PY{p}{)}
        \PY{k}{print}\PY{p}{(}\PY{n}{Pd}\PY{o}{/}\PY{l+m+mf}{1e6}\PY{p}{)}
\end{Verbatim}

    \begin{Verbatim}[commandchars=\\\{\}]
181.140442838
    \end{Verbatim}

    A maximum power of 181 MW.

    Now let's look how the power handling (ie the max power) evolves vs the
geometrical parameters $a$ and $b$. As seen above, the maximum power is:
\[
P_{max} 
= \frac{V^2_{max}}{2 Z_0}
\] with \[
    Z_0 = \frac{377}{2\pi}\sqrt{\frac{\mu_r}{\varepsilon_r}} \ln\left(\frac{b}{a}\right)
\] and \[
    V_{max} = E_{max} a \ln\left(\frac{b}{a}\right)
\]

    Thus, \[
P_{max} \approx \frac{E^2_{max} a^2}{120} \ln\left(\frac{b}{a}\right)
\] Or, rewriting in terms of $b/a$ : \[
P_{max} \approx \frac{E^2_{max} b^2}{120} \frac{ \ln\left(\frac{b}{a}\right) }{\left(b/a\right)^2}
\]

    \[
\frac{\partial P_{max}}{\partial a}
= 
0
\rightarrow
\frac{\partial}{\partial a}
\left( \frac{\ln(b/a)}{(b/a)^2}\right)
=
0
\] ie \[
2\ln(b/a) - 1 = 0
\] which is satisfied for \[
b/a=\exp(1/2)\approx 1.65
\]

    \begin{Verbatim}[commandchars=\\\{\}]
{\color{incolor}In [{\color{incolor}93}]:} \PY{n}{x} \PY{o}{=} \PY{n}{linspace}\PY{p}{(}\PY{l+m+mi}{1}\PY{p}{,} \PY{l+m+mi}{10}\PY{p}{,} \PY{l+m+mi}{1001}\PY{p}{)}
         \PY{n}{y} \PY{o}{=} \PY{n}{log}\PY{p}{(}\PY{n}{x}\PY{p}{)}\PY{o}{/}\PY{n}{x}\PY{o}{*}\PY{o}{*}\PY{l+m+mi}{2}
         \PY{n}{plot}\PY{p}{(}\PY{n}{x}\PY{p}{,} \PY{n}{y}\PY{p}{,} \PY{n}{lw}\PY{o}{=}\PY{l+m+mi}{2}\PY{p}{)}
         \PY{n}{xlabel}\PY{p}{(}\PY{l+s}{\PYZsq{}}\PY{l+s}{b/a}\PY{l+s}{\PYZsq{}}\PY{p}{)}
         \PY{n}{axvline}\PY{p}{(}\PY{n}{x}\PY{o}{=}\PY{n}{exp}\PY{p}{(}\PY{l+m+mi}{1}\PY{o}{/}\PY{l+m+mi}{2}\PY{p}{)}\PY{p}{,} \PY{n}{color}\PY{o}{=}\PY{l+s}{\PYZsq{}}\PY{l+s}{r}\PY{l+s}{\PYZsq{}}\PY{p}{)}
         \PY{n}{grid}\PY{p}{(}\PY{n+nb+bp}{True}\PY{p}{)}
         \PY{n}{title}\PY{p}{(}\PY{l+s}{\PYZsq{}}\PY{l+s}{\PYZdl{}}\PY{l+s}{\PYZbs{}}\PY{l+s}{ln(b/a) / (b/a)\PYZca{}2\PYZdl{}}\PY{l+s}{\PYZsq{}}\PY{p}{)}
         \PY{k}{print}\PY{p}{(}\PY{l+s}{\PYZsq{}}\PY{l+s}{The maximum is for x=}\PY{l+s}{\PYZsq{}}\PY{p}{,} \PY{n}{x}\PY{p}{[}\PY{n}{argmax}\PY{p}{(}\PY{n}{y}\PY{p}{)}\PY{p}{]}\PY{p}{)}
\end{Verbatim}

    \begin{Verbatim}[commandchars=\\\{\}]
The maximum is for x= 1.648
    \end{Verbatim}

    \begin{center}
    \adjustimage{max size={0.9\linewidth}{0.9\paperheight}}{IC-Hands-on_solutions_files/IC-Hands-on_solutions_17_1.png}
    \end{center}
    { \hspace*{\fill} \\}
    
    The second term is maximum at $b/a$ is around 1.65.

    \begin{Verbatim}[commandchars=\\\{\}]
{\color{incolor}In [{\color{incolor}7}]:} \PY{k}{def} \PY{n+nf}{coax\PYZus{}char\PYZus{}impedance}\PY{p}{(}\PY{n}{a}\PY{p}{,} \PY{n}{b}\PY{p}{,} \PY{n}{eps\PYZus{}r}\PY{o}{=}\PY{l+m+mi}{1}\PY{p}{,} \PY{n}{mu\PYZus{}r}\PY{o}{=}\PY{l+m+mi}{1}\PY{p}{)}\PY{p}{:}
            \PY{l+s+sd}{\PYZdq{}\PYZdq{}\PYZdq{}}
        \PY{l+s+sd}{    Returns the characteristic impedance of a coaxial line}
        \PY{l+s+sd}{    \PYZdq{}\PYZdq{}\PYZdq{}}
            \PY{k}{return} \PY{l+m+mi}{377}\PY{o}{/}\PY{p}{(}\PY{l+m+mi}{2}\PY{o}{*}\PY{n}{pi}\PY{p}{)}\PY{o}{*}\PY{n}{sqrt}\PY{p}{(}\PY{n}{mu\PYZus{}r}\PY{o}{/}\PY{n}{eps\PYZus{}r}\PY{p}{)}\PY{o}{*}\PY{n}{log}\PY{p}{(}\PY{n}{b}\PY{o}{/}\PY{n}{a}\PY{p}{)}
\end{Verbatim}

    This maximum corresponds to an impedance of :

    \begin{Verbatim}[commandchars=\\\{\}]
{\color{incolor}In [{\color{incolor}95}]:} \PY{n}{coax\PYZus{}char\PYZus{}impedance}\PY{p}{(}\PY{l+m+mi}{1}\PY{p}{,} \PY{l+m+mf}{1.65}\PY{p}{)}
\end{Verbatim}

            \begin{Verbatim}[commandchars=\\\{\}]
{\color{outcolor}Out[{\color{outcolor}95}]:} 30.047225143476474
\end{Verbatim}
        
    \section{4. Measurements ar low frequencies}

    Let's fix the frequency and vary the dimensions to see how evolve the
attenutation

    \[
\frac{\partial \alpha}{\partial a} = 0 
\rightarrow 
\frac{R_s}{2} \sqrt{\frac{\varepsilon}{\mu}} 
\frac{\partial }{\partial a}
\left( 
\frac{1/a+1/b}{\ln(b/a)}
\right)
=0
\]

\[
\rightarrow 1 + a/b - \ln\left(\frac{b}{a}\right) = 0
\]

    Let's see where this equation is satisfied :

    \begin{Verbatim}[commandchars=\\\{\}]
{\color{incolor}In [{\color{incolor}99}]:} \PY{k+kn}{from} \PY{n+nn}{scipy.optimize} \PY{k+kn}{import} \PY{n}{minimize}
         
         \PY{k}{def} \PY{n+nf}{fun}\PY{p}{(}\PY{n}{u}\PY{p}{)}\PY{p}{:}
             \PY{k}{return} \PY{n+nb}{abs}\PY{p}{(}\PY{l+m+mi}{1}\PY{o}{+}\PY{l+m+mi}{1}\PY{o}{/}\PY{n}{u}\PY{o}{\PYZhy{}}\PY{n}{log}\PY{p}{(}\PY{n}{u}\PY{p}{)}\PY{p}{)}
         
         \PY{n}{sol} \PY{o}{=} \PY{n}{minimize}\PY{p}{(}\PY{n}{fun}\PY{p}{,} \PY{n}{x0}\PY{o}{=}\PY{l+m+mi}{3}\PY{p}{)}
         \PY{k}{print}\PY{p}{(}\PY{n}{sol}\PY{o}{.}\PY{n}{x}\PY{p}{)}
\end{Verbatim}

    \begin{Verbatim}[commandchars=\\\{\}]
[ 3.59112147]
    \end{Verbatim}

    \begin{Verbatim}[commandchars=\\\{\}]
{\color{incolor}In [{\color{incolor}117}]:} \PY{n}{u} \PY{o}{=} \PY{n}{linspace}\PY{p}{(}\PY{l+m+mf}{0.1}\PY{p}{,} \PY{l+m+mi}{10}\PY{p}{,} \PY{l+m+mi}{501}\PY{p}{)}
          \PY{n}{test} \PY{o}{=} \PY{l+m+mi}{1}\PY{o}{+}\PY{l+m+mi}{1}\PY{o}{/}\PY{n}{u}\PY{o}{\PYZhy{}}\PY{n}{log}\PY{p}{(}\PY{n}{u}\PY{p}{)}
          \PY{n}{plot}\PY{p}{(}\PY{n}{u}\PY{p}{,} \PY{n}{test}\PY{p}{,} \PY{n}{lw}\PY{o}{=}\PY{l+m+mi}{2}\PY{p}{)}
          \PY{n}{xlabel}\PY{p}{(}\PY{l+s}{\PYZsq{}}\PY{l+s}{b/a}\PY{l+s}{\PYZsq{}}\PY{p}{)}
          \PY{n}{title}\PY{p}{(}\PY{l+s}{\PYZsq{}}\PY{l+s}{\PYZdl{}1+1/(b/a)\PYZhy{}ln(b/a)\PYZdl{}}\PY{l+s}{\PYZsq{}}\PY{p}{)}
          \PY{c}{\PYZsh{}u\PYZus{}0 = u[argmin(abs(test))]}
          \PY{n}{axhline}\PY{p}{(}\PY{n}{y}\PY{o}{=}\PY{l+m+mi}{0}\PY{p}{,} \PY{n}{color}\PY{o}{=}\PY{l+s}{\PYZsq{}}\PY{l+s}{k}\PY{l+s}{\PYZsq{}}\PY{p}{)}
          \PY{n}{grid}\PY{p}{(}\PY{p}{)}
          \PY{n}{axvline}\PY{p}{(}\PY{n}{x}\PY{o}{=}\PY{n}{sol}\PY{o}{.}\PY{n}{x}\PY{p}{,} \PY{n}{color}\PY{o}{=}\PY{l+s}{\PYZsq{}}\PY{l+s}{r}\PY{l+s}{\PYZsq{}}\PY{p}{)}
\end{Verbatim}

            \begin{Verbatim}[commandchars=\\\{\}]
{\color{outcolor}Out[{\color{outcolor}117}]:} <matplotlib.lines.Line2D at 0xbf1f780>
\end{Verbatim}
        
    \begin{center}
    \adjustimage{max size={0.9\linewidth}{0.9\paperheight}}{IC-Hands-on_solutions_files/IC-Hands-on_solutions_27_1.png}
    \end{center}
    { \hspace*{\fill} \\}
    
    The latter equation is satisfied for $b/a=3.6$, which gives an optimum
characteristic impedance of 77 Ohm:

    \begin{Verbatim}[commandchars=\\\{\}]
{\color{incolor}In [{\color{incolor}101}]:} \PY{n}{a} \PY{o}{=} \PY{l+m+mf}{200e\PYZhy{}3}
          \PY{n}{coax\PYZus{}char\PYZus{}impedance}\PY{p}{(}\PY{n}{a}\PY{p}{,} \PY{l+m+mf}{3.6}\PY{o}{*}\PY{n}{a}\PY{p}{)}
\end{Verbatim}

            \begin{Verbatim}[commandchars=\\\{\}]
{\color{outcolor}Out[{\color{outcolor}101}]:} 76.857841386182059
\end{Verbatim}
        
    \section{Optimum ?}

Let's plot together the attenuation and the power handling vs
characteristic impedance vs Z0

    \begin{Verbatim}[commandchars=\\\{\}]
{\color{incolor}In [{\color{incolor}118}]:} \PY{k}{def} \PY{n+nf}{coax\PYZus{}maximum\PYZus{}power}\PY{p}{(}\PY{n}{a}\PY{p}{,} \PY{n}{b}\PY{p}{,} \PY{n}{Emax}\PY{o}{=}\PY{l+m+mf}{3.0e6}\PY{p}{)}\PY{p}{:}
              \PY{c}{\PYZsh{} deduces the maximum electric field from voltage breakdown value}
              \PY{n}{Vp} \PY{o}{=} \PY{n}{Emax}\PY{o}{*}\PY{n}{a}\PY{o}{*}\PY{n}{log}\PY{p}{(}\PY{n}{b}\PY{o}{/}\PY{n}{a}\PY{p}{)}
              \PY{n}{Z0} \PY{o}{=} \PY{n}{coax\PYZus{}char\PYZus{}impedance}\PY{p}{(}\PY{n}{a}\PY{p}{,} \PY{n}{b}\PY{p}{)}
              \PY{k}{return} \PY{n}{Vp}\PY{o}{*}\PY{o}{*}\PY{l+m+mi}{2} \PY{o}{/} \PY{p}{(}\PY{l+m+mi}{2}\PY{o}{*}\PY{n}{Z0}\PY{p}{)}
\end{Verbatim}

    \begin{Verbatim}[commandchars=\\\{\}]
{\color{incolor}In [{\color{incolor}97}]:} \PY{k}{def} \PY{n+nf}{coax\PYZus{}alpha}\PY{p}{(}\PY{n}{a}\PY{p}{,} \PY{n}{b}\PY{p}{,} \PY{n}{f}\PY{p}{,} \PY{n}{sigma}\PY{o}{=}\PY{l+m+mf}{6e7}\PY{p}{)}\PY{p}{:}
             \PY{c}{\PYZsh{} sheet resistance [Ohm]}
             \PY{n}{Rs} \PY{o}{=} \PY{n}{sqrt}\PY{p}{(}\PY{l+m+mi}{2}\PY{o}{*}\PY{n}{pi}\PY{o}{*}\PY{n}{f}\PY{o}{*}\PY{n}{mu\PYZus{}0}\PY{o}{/}\PY{p}{(}\PY{l+m+mi}{2}\PY{o}{*}\PY{n}{sigma}\PY{p}{)}\PY{p}{)}
             \PY{c}{\PYZsh{} Resistance per meter [Ohm/m]}
             \PY{n}{R} \PY{o}{=} \PY{n}{Rs}\PY{o}{/}\PY{p}{(}\PY{l+m+mi}{2}\PY{o}{*}\PY{n}{pi}\PY{p}{)}\PY{o}{*}\PY{p}{(}\PY{l+m+mi}{1}\PY{o}{/}\PY{n}{a} \PY{o}{+} \PY{l+m+mi}{1}\PY{o}{/}\PY{n}{b}\PY{p}{)}
             \PY{c}{\PYZsh{} Characteristic Impedance [Ohm]}
             \PY{n}{Z0} \PY{o}{=} \PY{n}{coax\PYZus{}char\PYZus{}impedance}\PY{p}{(}\PY{n}{a}\PY{p}{,} \PY{n}{b}\PY{p}{)}
             \PY{c}{\PYZsh{} Attenuation [1/m]}
             \PY{k}{return} \PY{n}{R}\PY{o}{/}\PY{p}{(}\PY{l+m+mi}{2}\PY{o}{*}\PY{n}{Z0}\PY{p}{)}
\end{Verbatim}

    \begin{Verbatim}[commandchars=\\\{\}]
{\color{incolor}In [{\color{incolor}119}]:} \PY{n}{a} \PY{o}{=} \PY{n}{linspace}\PY{p}{(}\PY{l+m+mf}{20e\PYZhy{}3}\PY{p}{,} \PY{l+m+mf}{450e\PYZhy{}3}\PY{p}{)} \PY{c}{\PYZsh{} inner conductor radius}
          \PY{n}{b} \PY{o}{=} \PY{l+m+mf}{500e\PYZhy{}3}
          \PY{n}{f} \PY{o}{=} \PY{l+m+mf}{60e6}
          
          \PY{n}{Z0} \PY{o}{=} \PY{n}{coax\PYZus{}char\PYZus{}impedance}\PY{p}{(}\PY{n}{a}\PY{p}{,} \PY{n}{b}\PY{p}{)}
          
          \PY{n}{alpha} \PY{o}{=} \PY{n}{coax\PYZus{}alpha}\PY{p}{(}\PY{n}{a}\PY{p}{,} \PY{n}{b}\PY{p}{,} \PY{n}{f}\PY{p}{)}
          \PY{n}{alpha\PYZus{}norm} \PY{o}{=} \PY{n}{alpha}\PY{o}{/}\PY{n+nb}{min}\PY{p}{(}\PY{n}{alpha}\PY{p}{)}
          
          \PY{n}{max\PYZus{}pow} \PY{o}{=} \PY{n}{coax\PYZus{}maximum\PYZus{}power}\PY{p}{(}\PY{n}{a}\PY{p}{,} \PY{n}{b}\PY{p}{)}
          \PY{n}{max\PYZus{}pow\PYZus{}norm} \PY{o}{=} \PY{n}{max\PYZus{}pow}\PY{o}{/}\PY{n+nb}{max}\PY{p}{(}\PY{n}{max\PYZus{}pow}\PY{p}{)}
\end{Verbatim}

    \begin{Verbatim}[commandchars=\\\{\}]
{\color{incolor}In [{\color{incolor}149}]:} \PY{n}{semilogx}\PY{p}{(}\PY{n}{Z0}\PY{p}{,} \PY{n}{alpha\PYZus{}norm}\PY{p}{,} \PY{n}{lw}\PY{o}{=}\PY{l+m+mi}{2}\PY{p}{)}
          \PY{n}{semilogx}\PY{p}{(}\PY{n}{Z0}\PY{p}{,} \PY{n}{max\PYZus{}pow\PYZus{}norm}\PY{p}{,} \PY{l+s}{\PYZsq{}}\PY{l+s}{\PYZhy{}\PYZhy{}}\PY{l+s}{\PYZsq{}}\PY{p}{,} \PY{n}{lw}\PY{o}{=}\PY{l+m+mi}{2}\PY{p}{)}
          \PY{n}{legend}\PY{p}{(}\PY{p}{(}\PY{l+s}{\PYZsq{}}\PY{l+s}{Attenuation}\PY{l+s}{\PYZsq{}}\PY{p}{,} \PY{l+s}{\PYZsq{}}\PY{l+s}{Power handling}\PY{l+s}{\PYZsq{}}\PY{p}{)}\PY{p}{,} \PY{n}{fontsize}\PY{o}{=}\PY{l+m+mi}{14}\PY{p}{,} \PY{n}{loc}\PY{o}{=}\PY{l+s}{\PYZsq{}}\PY{l+s}{best}\PY{l+s}{\PYZsq{}}\PY{p}{)}
          
          \PY{n}{grid}\PY{p}{(}\PY{n+nb+bp}{True}\PY{p}{,} \PY{n}{which}\PY{o}{=}\PY{l+s}{\PYZsq{}}\PY{l+s}{major}\PY{l+s}{\PYZsq{}}\PY{p}{)}
          \PY{n}{grid}\PY{p}{(}\PY{n+nb+bp}{True}\PY{p}{,} \PY{n}{which}\PY{o}{=}\PY{l+s}{\PYZsq{}}\PY{l+s}{minor}\PY{l+s}{\PYZsq{}}\PY{p}{,} \PY{n}{axis}\PY{o}{=}\PY{l+s}{\PYZsq{}}\PY{l+s}{x}\PY{l+s}{\PYZsq{}}\PY{p}{)}
          \PY{n}{xlabel}\PY{p}{(}\PY{l+s}{\PYZsq{}}\PY{l+s}{\PYZdl{}Z\PYZus{}0\PYZdl{} [\PYZdl{}}\PY{l+s}{\PYZbs{}}\PY{l+s}{Omega\PYZdl{}]}\PY{l+s}{\PYZsq{}}\PY{p}{,} \PY{n}{fontsize}\PY{o}{=}\PY{l+m+mi}{14}\PY{p}{)}
          \PY{n}{ylabel}\PY{p}{(}\PY{l+s}{\PYZsq{}}\PY{l+s}{Normalized Values}\PY{l+s}{\PYZsq{}}\PY{p}{,} \PY{n}{fontsize}\PY{o}{=}\PY{l+m+mi}{14}\PY{p}{)}
          
          \PY{n}{axis}\PY{p}{(}\PY{p}{[}\PY{l+m+mi}{10}\PY{p}{,} \PY{l+m+mi}{200}\PY{p}{,} \PY{l+m+mi}{0}\PY{p}{,} \PY{l+m+mi}{2}\PY{p}{]}\PY{p}{)}
          \PY{n}{ax}\PY{o}{=}\PY{n}{gca}\PY{p}{(}\PY{p}{)}
          \PY{n}{ax}\PY{o}{.}\PY{n}{annotate}\PY{p}{(}\PY{l+s}{\PYZsq{}}\PY{l+s}{Maximum at \PYZdl{}30}\PY{l+s}{\PYZbs{}}\PY{l+s}{Omega\PYZdl{}}\PY{l+s}{\PYZsq{}}\PY{p}{,} \PY{n}{xy}\PY{o}{=}\PY{p}{(}\PY{l+m+mi}{30}\PY{p}{,}\PY{l+m+mi}{1}\PY{p}{)}\PY{p}{,} \PY{n}{xytext}\PY{o}{=}\PY{p}{(}\PY{l+m+mi}{15}\PY{p}{,}\PY{l+m+mf}{0.60}\PY{p}{)}\PY{p}{,} 
                      \PY{n}{color}\PY{o}{=}\PY{l+s}{\PYZsq{}}\PY{l+s}{g}\PY{l+s}{\PYZsq{}}\PY{p}{,} \PY{n}{arrowprops}\PY{o}{=}\PY{n+nb}{dict}\PY{p}{(}\PY{n}{facecolor}\PY{o}{=}\PY{l+s}{\PYZsq{}}\PY{l+s}{g}\PY{l+s}{\PYZsq{}}\PY{p}{,} \PY{n}{width}\PY{o}{=}\PY{l+m+mi}{5}\PY{p}{)}\PY{p}{,}
                      \PY{n}{fontsize}\PY{o}{=}\PY{l+m+mi}{14}\PY{p}{)}
          \PY{n}{ax}\PY{o}{.}\PY{n}{annotate}\PY{p}{(}\PY{l+s}{\PYZsq{}}\PY{l+s}{Maximum at \PYZdl{}77}\PY{l+s}{\PYZbs{}}\PY{l+s}{Omega\PYZdl{}}\PY{l+s}{\PYZsq{}}\PY{p}{,} \PY{n}{xy}\PY{o}{=}\PY{p}{(}\PY{l+m+mi}{77}\PY{p}{,}\PY{l+m+mi}{1}\PY{p}{)}\PY{p}{,} \PY{n}{xytext}\PY{o}{=}\PY{p}{(}\PY{l+m+mi}{50}\PY{p}{,}\PY{l+m+mf}{1.5}\PY{p}{)}\PY{p}{,} 
                      \PY{n}{color}\PY{o}{=}\PY{l+s}{\PYZsq{}}\PY{l+s}{b}\PY{l+s}{\PYZsq{}}\PY{p}{,} \PY{n}{arrowprops}\PY{o}{=}\PY{n+nb}{dict}\PY{p}{(}\PY{n}{facecolor}\PY{o}{=}\PY{l+s}{\PYZsq{}}\PY{l+s}{b}\PY{l+s}{\PYZsq{}}\PY{p}{,} \PY{n}{width}\PY{o}{=}\PY{l+m+mi}{5}\PY{p}{)}\PY{p}{,}
                      \PY{n}{fontsize}\PY{o}{=}\PY{l+m+mi}{14}\PY{p}{)}
\end{Verbatim}

            \begin{Verbatim}[commandchars=\\\{\}]
{\color{outcolor}Out[{\color{outcolor}149}]:} <matplotlib.text.Annotation at 0xf522390>
\end{Verbatim}
        
    \begin{center}
    \adjustimage{max size={0.9\linewidth}{0.9\paperheight}}{IC-Hands-on_solutions_files/IC-Hands-on_solutions_34_1.png}
    \end{center}
    { \hspace*{\fill} \\}
    
    \begin{Verbatim}[commandchars=\\\{\}]
{\color{incolor}In [{\color{incolor}}]:} 
\end{Verbatim}


    % Add a bibliography block to the postdoc
    
    
    
    \end{document}
